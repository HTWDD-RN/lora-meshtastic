\documentclass[12pt,a4paper]{article}
\usepackage[utf8]{inputenc}
\usepackage[german]{babel}
\usepackage{amssymb}
\usepackage{amsmath}
\usepackage{amsfonts}
\usepackage{tikz}
\usetikzlibrary{arrows ,automata ,positioning}
\usepackage{amssymb}
\usepackage{graphicx}
\usepackage{epstopdf}
\usepackage{tabto}
\usepackage[left=2cm,right=2cm,top=2cm,bottom=2cm]{geometry}
\author{Christian Grieß}
\begin{document}
\begin{titlepage}
\begin{figure}
	\centering \includegraphics[scale=1]{HTW_LOGO.png}
\end{figure}

	
	\centering
	{\scshape\LARGE hochschule für Technik und Wirtschaft Dresden \par}
	\vspace{2cm}
	{\scshape\Large Übertragung von Sensordaten mittels LoRaWAN\par}
	\vspace{0cm}
	{\scshape\Large Projektseminar \par}
	\vspace{1.5cm}
	{\huge\bfseries Katastrophennetz\par}
	{\huge\bfseries mithilfe von Meshtastic\par}
	\vspace{1.5cm}
	{\huge\bfseries {Dokumentation}\par}	
	\vspace{4cm}
	{\Large\itshape Christian Grieß\par}
	{\Large\itshape Göran Heinemann\par}
	{\Large\itshape Julian Meinking\par}
	\vfill
	unter Aufsicht von\par
	Prof. Dr.-Ing.~Jörg \textsc{Vogt}

	\vfill

% Bottom of the page
	{\large \today\par}
\end{titlepage}
\newpage
\tableofcontents

\newpage
\section{Aufgabenstellung}

Ziel dieses Projekts war das Experimentieren mit Meshtastic auf LoRa-fähigen Geräten.
Meshtastic ist ein Open-Source-Projekt, das es ermöglicht, ein Mesh-Netzwerk aufzubauen, das auf der LoRa-Technologie basiert. Es ist eine kostengünstige und energieeffiziente Möglichkeit, ein Netzwerk aufzubauen, das unabhängig von Internet und Mobilfunknetzen funktionieren kann.

\section{Fragestellung}

Ist Meshtastic als unabhängiges Kommunikations-Netzwerk für den Krisenfall im Raum Dresden geeignet?

\section{Technologie}
\subsection{Was ist LoRa?}

LoRa (von Long Range) ist eine proprietäre Funktechnologie im Besitz von Semtech. Sie ist für die Langstreckenübertragung (z.B. 10 km), schmalbandige Übertragung (gemessen in Kbps) und energiesparende Kommunikation konzipiert, hauptsächlich für Internet of Things (IoT)-Netzwerke. Dafür wird eine drahtlose Modulationstechnik, die aus der Chirp Spread Spectrum (CSS)-Technologie abgeleitet ist verwendet. Sie codiert Informationen auf Radiowellen mithilfe von Chirp-Impulsen! Die modulierte Übertragung von LoRa ist robust gegen Störungen und kann über große Entfernungen empfangen werden.

Es eignet sich ideal für Anwendungen, die kleine Datenmengen mit niedrigen Bitraten übertragen. Daten können über eine längere Reichweite übertragen werden im Vergleich zu Technologien wie Wlan, Bluetooth oder ZigBee. Diese Eigenschaften machen LoRa besonders geeignet für Sensoren und Aktoren, die im Niedrigenergiemodus arbeiten.

Außerdem arbeitet LoRa in einem lizenzfreien Sub-Gigahertz-Frequenzband (d.h. unter 1 GHz), aber die zu verwendenden Frequenzen variieren von Region zu Region aufgrund regulatorischer Anforderungen. Wenn Sie ein LoRa-Gerät kaufen, muss sichergestellt sein, dass das richtige Frequenzband unterstützt wird.

In Europa - 863–870MHz (normalerweise 868MHz).

\subsection{Warum LoRa?}

LoRa versucht die Lücke zwischen zwischen Kommunikationstechnologien wie WiFi, Bluetooth und LTE zu schließen.\\
\begin{figure}
	\includegraphics[scale=0.7]{bandwidth-vs-range.png}
\end{figure}

Semtech

Es ist für große Reichweite, kleine Bandbreite und Niedrigenergiekommunikation gemacht. Alles in allem also extrem nützlich für IoT Geräte. Einige Beispiele sind:

\begin{itemize}
	\item Wassersensoren in einer entfernten Umgebung (Grundwasser)
	\item Rauchwarnmelder
	\item Tierbeobachtung
	\item Verbrauchsmessungen bei Endkunden (Gas, Strom)
	\item Wetterstationen die nur ab und zu Informationen übertragen
\end{itemize}

\subsection{LoRa und LoRaWAN}

LoRaWAN ist über LoRa angesiedelt und definiert das Kommunikationsprotokoll und die Systemarchitektur.

Es ist wichtig zu verstehen, dass es möglich ist LoRa ohne LoRaWAN zu benutzen. Andere LoRa basierte Netzwerke sind Helium, The Things Network Disaster.radio und, was wir weiter betrachten werden, Meshtastic.

\subsection{Meshtastic}
Wie im vorherigen Absatz erwähnt, baut Meshtastic auf LoRa auf und schafft ein dezentralisiertes Mesh-Netzwerk.

Es bringt folgende Eigenschaften mit sich:
\begin{itemize}
	\item Verschlüsselte und Textbasierte Kommunikation
	\item Plattformunabhängig
	\begin{itemize}
		\item Computer (unabhänging vom Betriebssystem)
		\item Android (dedizierte Chat-App)
		\item iOS (dedizierte Chat-App)
	\end{itemize}
	\item Dezentralisiert
	\item Geringer Stromverbrauch
	\item Optionales Standort teilen
	\item Open-source
\end{itemize}

Anders als traditionelle Mobilfunknetzwerke, verbindet sich jedes Endnutzergerät mit einem LoRa Radio und alle LoRa Radios, welche Meshtastic nutzen, können Nachrichten, selbst wenn die Radios nicht im gleichen Mesh sind, weiterleiten. Das passiert so lange, bis die Nachricht Ihr Ziel erreicht oder die voreingestellten “Hops” ausgeschöpft werden.


\section{Werkzeuge}
\begin{description}


    \item [Programmiersprache]\tab Python
    \item [Open-Source Bibliotheken]\tab Keras \newline \tab \tab Tensorflow
 
 
\end{description}

\section{Was ist Maschinelles Lernen?}
Maschinelles Lernen ist das Fachgebiet, das Computern die Fähigkeit zu lernen verleiht ohne explizit programmiert zu werden
\begin{flushright} - Arthur Samuel 1959  \end{flushright}

\section{Was sind Rekurrente Neurale Netzwerke?}
Ein Rekurrentes Neurales Netzwerk hat im gegensatz zu anderen neuralen Netzen (z.B. Feedforward Neural Network) Rückwärts gerichtete Neuronen. 
Die einfachste Variante besteht aus nur einem Neuron, dass Eingaben erhält, eine Ausgabe produziert und diese Ausgabe wieder an sich selbst schickt.\newline

\begin{tikzpicture}
  \path 
  
  	(-1,2) node (y0){y}
    	(5,2) node (y1){$y_{(t-2)}$}
	(9,2) node (y2){$y_{(t-1)}$}
	(13,2) node (y3){$y_{(t)}$}
	
	(-1,-2) node (x0) {x}
	(5,-2) node (x1) {$x_{(t-2)}$}  	
	(9,-2) node (x2) {$x_{(t-1)}$}
	(13,-2) node (x3) {$x_{(t)}$}
	
	(13,-3) node (t) {$t$}
	
    	(-1,0) node[circle split,draw,double,fill=red!20](z0)
  		{
    		$Ausgabe$
    		\nodepart{lower}
    		$Eingabe$
  		}
  	(5,0) node[circle split,draw,double,fill=red!20](z1)
  		{
    		$Ausgabe$
    		\nodepart{lower}
    		$Eingabe$
  		}
	(9,0) node[circle split,draw,double,fill=red!20](z2)
  		{
		$Ausgabe$
    		\nodepart{lower}
    		$Eingabe$
  		}
	(13,0) node[circle split,draw,double,fill=red!20](z3)
  		{
    		$Ausgabe$
    		\nodepart{lower}
    		$Eingabe$
  		};
  \draw[-latex] (3,0) -- (z1.south west);

  \draw[-latex] (z1.north east) -- (z2.south west);
  \draw[-latex] (z2.north east) -- (z3.south west);
  
  \draw[-latex] (z3.north east) -- (15,0);
	
  \draw[->, black] (z0) -- (y0);
  \draw[->, black] (x0) -- (z0);  
  
  \draw[->, black] (z1) -- (y1);
  \draw[->, black] (x1) -- (z1);

  \draw[->, black] (z2) -- (y2);
  \draw[->, black] (x2) -- (z2);
  
  \draw[->, black] (z3) -- (y3);
  \draw[->, black] (x3) -- (z3);	
  
  \draw[->, black] (5,-3) -- (t);
 
  
  \end{tikzpicture}
 
Einzelnes Neuron links       \tab    entlang der Zeitachse aufgerollt rechts


Durch die Rückkopplungen können lokale Optima entstehen, die zu unübersichtlichen Netzzuständen führen. Rekurrente neuronale Netze sind nicht in der Lage, weit entfernte oder zurückliegende Informationen miteinander zu verknüpfen. Netze mit sogenannten LSTM-Zellen (Long Short Term Memory) lösen dieses Problem, indem die Zellen neben einem Eingang und einem Ausgang ein Merk- und Vergesstor besitzen. Es entsteht eine Art Kurzzeitgedächtnis, das Informationen relativ lange vorhalten kann und sich an früher gemachte Erfahrungen erinnert.

Rekurrente neuronale Netze eignen sich im Gegensatz zu den Feedforward-Netzen nicht nur für die einfache Mustererkennung. Sie sind in der Lage, komplexere Sequenzen zu verarbeiten. 

\subsection{ML-Modell}
$<Grafik>$ \newline
Grobe Beschreibung, was, warum, in welcher Reihenfolge gemacht wird.
\subsubsection{GRU-Layer}
Why GRU (Vorteile, Nachteile) \newline
Verallgemeinerte(vereinfachte) Grafik (Update Gate, Reset Gate, Current Memory)\newline
Grafik Update Gate\newline
etc.
\subsubsection{LSMT-Layer}
Placeholder
\subsubsection{Dense-Layer}
Placeholder
\subsubsection{Embedding-Layer}
Placeholder
\section{Codevorstellung und Beschreibung$(Einzelschritte?)$}
Placeholder incl. Auswertungsgrafiken
\newpage

\section{Discussion}
\section{Summary}
\section{References}
\end{document}
